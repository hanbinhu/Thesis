%# -*- coding: utf-8-unix -*-
\begin{thanks}

\ifsjtu@review\relax\else

转眼之间,在交大的6年半时光即将过去,我也即将从交大毕业。六年多的时间带给我的绝不仅仅是只是学业与专业技能方面的长进,更多的是对这座校园、这里的人的留恋与不舍。从刚刚高中毕业,懵懂无知的少年踏入校园;想如今,带着这沉甸甸的留恋就要离开这座百年学府,不惊令人感慨:岁月如斯。

首先我最要感谢的是我研究生导师施国勇教授。研究生期间,施老师一直是我的榜样。施老师平日里兢兢业业,专心学术的精神令我感动。读论文,做研究,写代码,与学生讨论课题研究进展,这些看似稀松平常的事早已成为施老师生活不可或缺的一部分,更是他作为一个学者的乐趣所在。记得曾听施老师说起:“我发现了一个问题,尝试去研究它,钻研它,然后解决了这个问题,这会给我带来至高的乐趣。”丰富的研究经验与渊博的知识让施老师总是能在我研究遇到困难的时候给我指出问题所在,帮助我走得更顺利。更多的时候,施老师更是身体力行,亲自与我们一起开展研究,开发代码。在国内浮躁的环境中,能遇到这样的一位静心科研的导师是我最大的幸运。平日里,施老师也不忘在朋友圈中分享自己的书法作品,与我们一起去KTV唱歌,陶冶性情,让我们认识到了施老师除学术研究之外的在文学、音乐上的造诣。希望日后能有机会出国留学深造,也能成为施老师这样醉心学术有性情的人。

其次我要感谢MSDA实验室各位同学,如果不是你们的陪伴,我很难有今天的成绩。记得刚来到实验室,朱彦学姐手把手带着我开展科研;程建东学长高超的专业技巧令我钦佩。陈静学长和张爱林学姐经常与我一起讨论学术问题。最近,学姐更是喜迎贵子,祝学姐的宝宝健康快乐茁壮成长。在Synopsys实习期间,陈家俊学长和李骥学长也给予了我不少帮助。几年来,实验室也增添了不少新生力量,聪明开朗的顾艳捷,认真努力的邓舒文,很高兴能与你们一起共事。也祝愿新进实验室的李博、何津、郭海天能早日撑起实验室的一片天,让实验室发展壮大。

此外我还感谢我的学院中的各位同学们。感谢我的室友秦小波同学平日里点点滴滴对我的关心与照顾。感谢刘帅和章裕同学每周末都陪我去羽毛球场挥洒汗水。感谢祝希同学与我一同在Synopsys实习期间度过的愉快时光。感谢胡锁平和潘布堃同学总与我嘻笑怒骂,平日生活因为你们而增色不少。感谢钱晨同学总是在奖学金的事宜上热心地帮助我。感谢学院中的每一位同学,感谢你们的平时点点滴滴,是缘分让我们相聚在一起。

最后我要感谢我的父母。家永远是我温暖的港湾,无论在外经受怎样的挫折挑战,你们总是笑脸相迎,包容我的点点滴滴。

即将离开交大,踏向远方,但我不会忘记这里是我的第二故乡,希望每个交大的同学都能展翅高飞,奔向光明的未来。

\fi

\end{thanks}
