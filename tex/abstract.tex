%# -*- coding: utf-8-unix -*-
%%==================================================
%% abstract.tex for SJTU Master Thesis
%%==================================================

\begin{abstract}

如今,模拟集成电路设计仍然在很大程度上依赖于手工分析,这严重阻碍了SoC的整体设计进程。作为模拟集成电路设计中不可或缺的一环,可靠的宏模型缺乏系统化自动化的设计流程。本文讨论了电路拓扑结构和双图决策树(Graph Pair Decision Diagram, GPDD)符号化方法之间的联系,并提出拓扑简化的想法,开发了一套行之有效的自动化低阶模型生成算法。这套算法生成的简化符号化模型与原始电路无论在性能上,还是拓扑结构上都有很好的匹配。算法的有效性和鲁棒性得到了不同电路结构、不同电路元件尺寸的广泛验证。

本论文将提出的模型生成方法应用至两方面。其一是通过在简化电路模型添加电流限制,来实现自动化构造大信号分析中用到的时域模型,从而进一步分析其转换速率。另一个应用是使用符号化多端口构造方法下的敏感度计算用于分析电路的共模抑制比和电源抑制比。本文给出了相关的测试结果和算法性能结果。

\keywords{\large 低阶模型生成 \quad 拓扑简化 \quad 双图决策树(GPDD) \quad 时域模型 \quad 共模抑制比(CMRR) \quad 电源抑制比(PSRR) \quad 符号化分析(Symbolic analysis)}
\end{abstract}

\begin{englishabstract}

Nowadays, analog integrated circuit (IC) design still depends heavily on manual analysis, which greatly impedes the entire SoC design process. As one indispensable part of analog IC design, reliable macromodel for analysis lacks systematic and automatic generation procedure. This thesis discusses the relation between circuit topology and a symbolic method call Graph-Pair Decision Diagram (GPDD). The topological simplification is proposed with an efficient symbolic algorithm for automatic low-order model generation. According to the experimental results, the simplified model matches the original circuits in performance and topology to good agreement. The validity and robustness of proposed method have been verified by circuits with various topologies and sizings extensively.

The proposed model generation method has been applied to two scenarios. The first problem is to automatically construct the time-domain model for large-signal analysis, and the typical slew-settling behavior can be captured by the proposed model. The other one is to analyze common mode rejection ratio (CMRR) and power supply rejection ratio (PSRR) with symbolic calculation and sensitivity analysis under multi-port construction. In the thesis, relevant test results and algorithm performance are provided.


\englishkeywords{\large Low-order model generation, topological simplification, graph-Pair decision diagram (GPDD), time-domain model, common mode rejection ratio (CMRR), power supply rejection ratio (PSRR), Symbolic Analysis}
\end{englishabstract}

