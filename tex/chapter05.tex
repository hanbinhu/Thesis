%# -*- coding: utf-8-unix -*-
\chapter{结束语}
\label{chap:con}

\section{主要工作与创新点}
\label{sec:con:job}

本文主要对电路的简化小信号模型生成算法进行了介绍,并尝试将其中的内容应用到时域模型的生成。并对CMRR以及PSRR分析进行了探索,为进一步的研究提供基础。下面将针对这三方面做逐一总结。

低阶符号化电路模型的自动生成是本文的开创性工作。过去的符号化简化工作往往仅针对电路的传输函数表达式,而不是针对电路的拓扑结构。相比的单纯的符号化公式,本文生成的简化的电路拓扑结构可以带给电路设计工程师更多电路的信息。本文通过挖掘电路拓扑结构与元件取值的关系引出了符号化构造方法GPDD与电路拓扑结构的关系。因为这种关系,GPDD承担了算法的主要计算职责。我们给出了符号化简化的主要流程,并特别提出了用于电路简化的元件重要性的概念,同时辅以元件的预处理方法,可以自动化系统化地生成可供人理解的小信号电路。通过对特殊情况的讨论,提高了算法的可靠性,避免程序中途中止。文中给出详细的测试结果以说明该算法的有效性与鲁棒性,在针对各种电路拓扑和电路参数尺寸的情况下,本文的算法都可以给出相应的符号化小信号模型。这一工作将大大方便模拟电路工程师对电路的分析工作,因为模型的自动生成,工程师可以直接定位电路的关键问题所在。另外在模拟电路IP化的进程中,本文中提出的方法可以尝试作为模拟电路的高层模型,以在不丢失大量精度的前提下,加速系统级的仿真的速度。

时域电路模型的生成通过在自动生成的简化小信号电路加入电流饱和限制以形成可以进行大信号分析的宏模型。本文创新地提出了对电路中多级均加入饱和电流限制的方法,并给出了多种非线性函数,用以形成饱和限制电流。这一模型生成方法目前的测试表明,已存在其可适用的情况;如需广泛使用,需进一步优化模型。

在如今电源电压日益降低的年代,CMRR和PSRR成为了衡量电路的抗共模噪声和电源干扰的能力的重要指标。本文采用了多端口的GPDD构造方式对电路的差模增益、共模抑制比和电源抑制比进行了分析。我们证明了多端口构造的适应性条件,为同时分析多个端口的电路的提供了良好的基础。通过GPDD由于节点共享造成的高效性,为同一时间分析三种电路参数提供了可能。本文还通过符号化敏感度求值等方法计算得到了电路优化方向,以同时优化多个电路参数,并更好进行折衷。进一步将来可以尝试多端口的降阶模型生成方法。

\section{后续研究方向}
\label{sec:con:future}

本文中所涉及的多个话题均有可以进一步研究的价值,如以下可能的研究方向:

\begin{enumerate}[label=\emph{\alph*})]
	\item 降阶模型生成中的元件重要性仍有许多可以挖掘的地方,如在重要性计算中单纯考虑运放的CMRR或者PSRR,或许可以得到用于分析CMRR和PSRR的对应的小信号电路。若同时结合多端口的构造策略,尝试得到能同时考虑多个运放指标的电路模型。这一方面实验室中已由本科同学进一步开展研究,并讨论其在电路失配情况下的应用。
	\item CMRR和PSRR的自动优化流程可以在考虑敏感度的情况下,通过模拟退火等多种优化方法尝试考虑多个电路性能指标的模拟集成电路的自动优化流程。
	\item 多端口的构造方式可以应用电路的噪声分析中,因为考虑电路噪声时,一般都考虑某一节点上的噪声,但整个电路中可能有多处噪声源,这正适应于本文提出的构造哦啊条件。
	\item 分析时域模型的进一步拓展可能性,尝试从理论上对生成时域模型进行分析。
\end{enumerate}